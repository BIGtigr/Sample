\documentclass{article}
\usepackage[margin=1in]{geometry}
\usepackage{natbib}
\usepackage{amsmath}
\usepackage{color}
\usepackage[procnames]{listings}
\usepackage{textcomp}
\usepackage{setspace}
\usepackage{palatino}

\definecolor{gray}{gray}{0.2}
\definecolor{green}{rgb}{0,0.5,0}
\definecolor{lightgreen}{rgb}{0,0.7,0}
\definecolor{purple}{rgb}{0.5,0,0.5}
\definecolor{darkred}{rgb}{0.5,0,0}
\definecolor{orange}{rgb}{1,0.3,0}
\definecolor{medblue}{rgb}{0, 0.5, 1}
\definecolor{lightgray}{rgb}{0.98, 0.98, 0.98}
\definecolor{pink}{rgb}{0.8, 0, 0.8}



\usepackage{hyperref}
\hypersetup{
	colorlinks=true, %set true if you want colored links
	linktoc=all,     %set to all if you want both sections and subsections linked
	linkcolor=blue,  %choose some color if you want links to stand out
	urlcolor=blue
}



\lstset{
		language=python,
		frame=leftline,
		backgroundcolor=\color{lightgray},
		numbers=left,
		numberstyle=\scriptsize,
		basicstyle=\ttfamily\scriptsize\setstretch{1.2},
		stringstyle=\color{red},
		showstringspaces=false,
		alsoletter={1234567890},
		otherkeywords={\ , \}, \{},
		breaklines=true,
		float,
		keywordstyle=\ttfamily,
		emph={access,and,as,break,class,continue,def,del,elif,else,%
			except,exec,finally,for,from,global,if,import,in,is,%
			lambda,not,or,pass,print,raise,return,try,while,assert},
		emphstyle=\color{medblue}\bfseries,
		emph={[2]self},
		emphstyle=[2]\color{gray},	
		emph={[4]ArithmeticError,AssertionError,AttributeError,BaseException,%
			DeprecationWarning,EOFError,Ellipsis,EnvironmentError,Exception,%
			False,FloatingPointError,FutureWarning,GeneratorExit,IOError,%
			ImportError,ImportWarning,IndentationError,IndexError,KeyError,%
			KeyboardInterrupt,LookupError,MemoryError,NameError,None,%
			NotImplemented,NotImplementedError,OSError,OverflowError,%
			PendingDeprecationWarning,ReferenceError,RuntimeError,RuntimeWarning,%
			StandardError,StopIteration,SyntaxError,SyntaxWarning,SystemError,%
			SystemExit,TabError,True,TypeError,UnboundLocalError,UnicodeDecodeError,%
			UnicodeEncodeError,UnicodeError,UnicodeTranslateError,UnicodeWarning,%
			UserWarning,ValueError,Warning,ZeroDivisionError,abs,all,any,apply,%
			basestring,bool,buffer,callable,chr,classmethod,cmp,coerce,compile,%
			complex,copyright,credits,delattr,dict,dir,divmod,enumerate,eval,%
			execfile,exit,filter,float,frozenset,getattr,globals,hasattr,%
			hash,help,hex,id,input,int,intern,isinstance,issubclass,iter,len,%
			license,list,locals,long,map,max,min,object,oct,open,ord,pow,property,%
			quit,range,raw_input,reduce,reload,repr,reversed,round,set,setattr,%
			slice,sorted,staticmethod,str,sum,super,tuple,unichr,unicode,%
			vars,xrange,zip},
		emphstyle=[4]\color{purple}\bfseries,
		emph={[5]construct_model,assign_name,num_classes,codon_model,read_tree,print_tree,%
			branch_het,site_het,construct_frequencies,Site,Evolver,Genetics,%
			MatrixBuilder,aminoAcid_Matrix,nucleotide_Matrix,mechCodon_Matrix,%
			mutSel_Matrix,ECM_Matrix,EvoModels,Model,CodonModel,Tree,Partition,%
			StateFrequencies,EqualFrequencies,RandomFrequencies,CustomFrequencies,%
			ReadFrequencies,EmpiricalModelFrequencies},
		emphstyle=[5]\color{green}\bfseries,		
		upquote=true,
		morecomment=[s][\color{darkred}]{"""}{"""},
		morecomment=[s][\color{darkred}]{'''}{'''},
		commentstyle=\color{magenta}\slshape,
		literate={>>>}{\textbf{\textcolor{orange}{>{>}>}}}3%
		{...}{{\textcolor{gray}{...}}}3,
		procnamekeys={def,class},
		procnamestyle=\color{green}\textbf,
		tabsize=4	
}


% Python for external files
\newcommand\pythonexternal[2][]{{
		\lstinputlisting[#1]{#2}}}

% Python for inline
\newcommand\pythoninline[1]{{\lstinline!#1!}}



\begin{document}


\title{User manual for pyvolve v1.0}
\author{Stephanie J. Spielman}
\date{}
\maketitle

\tableofcontents

\section{Introduction}

Pyvolve (pronouced ``pie-volve'') is an open-source python module for simulating genetic data along a phylogeny according to Markov models of sequence evolution. The module is available for download on \href{https://github.com/sjspielman/pyvolve/releases}{github} (and see \href{http://sjspielman.org/pyvolve/}{here} for API documentation). Note that pyvolve has several dependencies, including \href{http://biopython.org/wiki/Download}{BioPython}, \href{http://www.scipy.org/install.html}{NumPy}, and \href{http://www.scipy.org/install.html}{SciPy}. These modules must be properly installed and in your python path for pyvolve to work properly. Please file any and all bug reports on the github repository \href{https://github.com/sjspielman/pyvolve/issues}{Issues} section.

Pyvolve is written such that it can be seemlessly incorporated into your python pipelines without having to interface with external software platforms. However, please note that for extremely large ($>$1000 taxa) and/or extremelely heterogenous simulations (e.g. where each site evolves according to a unique evolutionary model), pyvolve may be quite slow and thus may take several minutes to run. Faster sequence simulators you may find useful include (but are certainly not limited to!) \href{http://abacus.gene.ucl.ac.uk/software/indelible/}{Indelible} \cite{Fletcher2009} and  \href{http://bioinfolab.unl.edu/~cstrope/iSG/}{indel-Seq-Gen} \cite{Strope2007}. 

Pyvolve supports a variety of evolutionary models, including the following:
\begin{itemize}
	\item Nucleotide Models 
	\begin{itemize}
		\item Generalized time-reversible model \cite{GTR} and all nested variants
	\end{itemize}
	\item Amino-acid exchangeability models 
	\begin{itemize}
		\item JTT \cite{JTT}, WAG \cite{WAG}, and LG \cite{LG}
	\end{itemize}
	\item Codon models
	\begin{itemize}
		\item Mechanistic ($dN/dS$) models (MG-style \cite{MG94} and GY-style \cite{GY94})
		\item Empirical codon model \cite{ECM}
	\end{itemize}
	\item Mutation-selection models
	\begin{itemize}
		\item Halpern-Bruno model \cite{HB98}, implemented for codons and nucleotides
	\end{itemize}
\end{itemize}
Note that it is also possible to specify custom matrices (detailed in section~\ref{sec:custom} below). Both site-wise and temporal (branch) heterogeneity are supported. Sequences are simulated accordingy to standard methods \cite{Yang2006}.



\section{Basic Usage}

Similar to other simulation platforms, pyvolve evolves sequences in groups of \textbf{partitions}. Each partition has an associated size and model (or set of models, if branch heterogeneity is desired). All partitions will evolve according to the same phylogeny; if you wish to have each partition evolve according to a distinct phylogeny, I recommend performing several simulations and then merging the resulting alignments in the post-processing stage. 

Pseudocode for a simple simulation is given below.

\begin{lstlisting}
# Import the pyvolve module
import pyvolve

# Read in tree along which pyvolve should simulate
my_tree = pyvolve.read_tree(file = 'file_with_tree_for_simulating.tre')

# Define and construct evolutionary models
my_model = pyvolve.Model(<model_type>, <custom_model_parameters>)
my_model.construct_model()

# Define partitions
my_partition = pyvolve.Partition(models = my_model, size = 100)

# Evolve partitions with the callable Evolver() class
pyvolve.Evolver(tree = my_tree, partitions = my_partition)()
\end{lstlisting}

By default, sequences will be output to a fasta-formatted file called ``simulated\_alignment.fasta''. Two additional tab-delimited files, called ``site\_rates.txt'' and ``site\_rates\_info.txt'' are also output. These files provide useful information when heterogeneity (either site or branch) is implemented. The former file indicates to which partition and rate category (if no rate heterogeneity specified, these values will all be 1) each site belongs, and the latter file provides more specific information about each rate category, in particular its associated partition, probability, and value. To suppress creation of either file, or to change the file name, use the respective arguments \texttt{ratefile} and \texttt{infofile} when initializing an \texttt{Evolver} instance. Providing a value of either None or False will suppress file creation, or simply provide a string giving the desired output file name. To change the outputed alignment file name, set the argument \texttt{seqfile} (similar, providing False/None will suppress file creation). The sequence file format can also be changed with the argument \texttt{seqfmt}. Biopython is used to write sequence files, so consult the Biopython manual for available formats.

$<$model\_type$>$ is the type of model matrix. $<$custom\_model\_parameters$>$ is a \textit{dictionary} of parameters for your chosen model. See below for available model types and associated parameter keys.

\subsection{Nucleotide Models}

\begin{equation}
q_{ij} = \mu_{ij} \pi_j
\end{equation}

Parameters include mutation rates in a dictionary with keys giving the two nucleotides. All models here symmetric, so AT=TA.
Alternatively, you may provide kappa. Other rates will be 1


\begin{lstlisting}
# Import the pyvolve module
import pyvolve

# Read in tree along which pyvolve should simulate
my_tree = pyvolve.read_tree(file = 'file_with_tree_for_simulating.tre')

# Define and construct evolutionary models
nucleotide_rates = {'AC':0.5, 'AG':0.2, 'AT':0.3, 'CG':1.5, 'CT':1.6, 'GT':3.4} # symmetric. all unspecified are set to 1.
my_model = pyvolve.Model('nucleotide', nucleotide_rates)
my_model.construct_model()

# Define partitions
my_partition = pyvolve.Partition(models = my_model, size = 100)

# Evolve partitions with the callable Evolver() class
pyvolve.Evolver(tree = my_tree, partitions = my_partition)()
\end{lstlisting}


\subsection{Amino-acid models}

\begin{equation}
q_{ij} = r_{ij} \pi_j
\end{equation}


Only parameters here are state frequencies

\subsection{Mechanistic codon models}


\begin{equation}\label{eq:GY94}
q_{ij} = \left\{ 
\begin{array}{rl}
\mu_{o_it_j} \pi_j \alpha      & \text{synonymous change} \\
\mu_{o_it_j} \pi_j \beta       & \text{nonsynonymous change} \\
0                    & \text{multiple nucleotide changes} \\             
\end{array} \right.,
\end{equation}


MG-style matrix elements, for a the substitution from codon $i$ to $j$, are generally given by 
\begin{equation}\label{eq:MGstyle}
q_{ij} = \left\{ 
\begin{array}{rl}
\mu_{o_it_j}\pi_{t_j} \alpha         &\text{synonymous change} \\
\mu_{o_it_j}\pi_{t_j} \beta  &\text{nonsynonymous change} \\
0                             &\text{multiple nucleotide changes}       
\end{array} \right. ,
\end{equation}


Mutation parameters and either omega or alpha/beta.

\subsection{Empirical codon models}

\begin{equation}\label{eq:ecmrest}
q_{ij} = \left\{ 
\begin{array}{rl}
s^\ast_{ij} \pi_j \kappa(i,j) \alpha      &\text{synonymous change} \\
s^\ast_{ij} \pi_j \kappa(i,j) \beta       &\text{nonsynonymous change} \\
\end{array} \right.,
\end{equation}
Further, rest and unrest. The $\kappa(i,j)$ parameter is defined in their paper...


\subsection{Mutation-selection models}
\begin{equation}
q_{ij} = \left\{ 
\begin{array}{rl}
	\mu_{ij} \frac{S_{ij}}{1-1/S_{ij}} &\text{single nucleotide change} \\\\
	0                                  &\text{multiple nucleotide changes} \\             
\end{array} \right.,
\end{equation} where $S_ij$ is the scaled selection coefficient. These models may be defined in 1 of 2 ways: with state frequencies or with fitness values. State frequencies should be specified with the key 'state\_freqs' or with scaled fitness values ($F = 2Nf$) with the key 'fitness'. Either way you'll need mutation parameters.
Your fitness vector should be a list of 4 (for nucleotide models) or 20/61 for codon models. If you give 20, they are assigned exactly to codon.
pyvolve will know if you want this in nucleotide or codon space, you just need to say MutSel when defining Model()


\section{Site-wise heterogeneity}
\subsection{Nucleotide and amino-acid models}
\subsection{Codon models}
\subsection{Mutation-selection models}
\section{Temporal heterogeneity}
\section{Building a vector of stationary frequencies}\label{sec:freqs}
\section{Matrix scaling options}\label{sec:scaling}
\section{Using custom rate matrices}\label{sec:custom}

For this, I'll need to have a matrix argument in model definition, and some sanity checking on a provided matrix will have to be done.



\noindent This is my first python example:

\pythonexternal{script.py}












\bibliographystyle{plain}
\bibliography{citations}





\end{document}